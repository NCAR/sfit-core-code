\documentclass[a4paper]{article}
\usepackage{amsmath}
\usepackage{xcolor}

\begin{document}
\section{AVK $>$ 1}

If the AVK is greater than one, it does not mean, it is getting more information from the measurement than when the AVK is $1$, but the
measurement looses influcence again. This can be seen when reordering using the equation 

\begin{equation}
  \hat{x} = (I - A) x_a + A x_T
\end{equation}

where $x_T$ is the true variable and $\hat{x}$ is the derived estimation of the quantity $x$. Clearly, if the AVK $A$ is the unity matrix, $\hat{x}$ is solely determined by the measurement. However, if $A$ is the scaled (scale $>1$) unity matrix, the estimated result is only partially determined by the measurement. For getting a really large $A$ it has only 50 \% influence.


\section{Logarithmic state vector}

SFIT4 has the possibility to work on a logarithmic state vector. The internal state vector won't be the ratio $\frac{\hat{x}}{x_A}$ but instead $x = ln(x)$.

The state vector will be reverted to VMR in the result, but the saved AVK is still for the state vector in the
$ln(x)$-form, denoted here by $A'$.

In order to recalculate the AVK for the linear state (VMR). The state vector is in the form $x' = g(x) = ln(x)$, where $x$ is the profile vector in VMR units. The forward model can be written as

\begin{equation}
  y = F(x) = F(g^{-1}(x'))  
\end{equation}

The weighting function matrix, $K$, is the derivative of the forward model $F$ 

\begin{equation}
  K = \frac{\partial F}{\partial x} \stackrel{\mathrm{chain rule}}{=}
  \frac{\partial F}{\partial g^{-1}}\frac{\partial g^{-1}}{\partial x} = K'  \frac{\partial g^{-1}}{\partial x}
\end{equation}

Using the fact that the differential of the inverse of a function $f$ can be written as 

\begin{equation}
  \frac{\partial f^{-1}}{\partial x} = \frac{1}{\frac{\partial f}{\partial x}}
\end{equation}

we obtain using $g(x) = ln(x)$ and $\frac{\partial ln(x)}{\partial x} = \frac{1}{x}$

\begin{equation}
  K = K' \frac{1}{\frac{\partial g}{\partial x}} = K' x
\end{equation}

Hence, to calculate the VMR averaging kernel from the logarithmic averaging kernel,  using $A = D K$ and $A' = D K'$ we get

\begin{align*}
A &= DK = x^{-1} A' x\\
A &= D K = \textcolor{red}{D K' x = A' x}
%x^{-1} D K  = x^{-1} A
\end{align*}

\emph{Wie komme ich auf den Zusammenhang zwischen A und A' ?}

\emph{Shouldn't D be G to conform with Rodgers notation?}




\end{document}
