\documentclass[a4paper]{article}

\begin{document}

\section{Empirical apodisation and phase function}

\subsection{sfit4.ctl}

The empirical apodisation and phase deals with ILS errors due to
misalignemt of the instrument.

To switch it on set the following flags in the fw section:
\begin{description}
\item [fw.apod\_fcn] 
  fw.apod\_function = T\\
  fw.apod\_function.type = 0,1,2,3 or 4\\
  fw.apod\_function.order = 1,2,3,\ldots
\item [fw.phase\_fcn] 
  fw.phase\_function = T\\
  fw.phase\_function.type = 0,1,2 or 4 (3 does not exist)\\
  fw.phase\_function.order = 0, 1,2,3,\ldots\\
\item[rt.phase\_fcn = T ] retrieval of the empirical phase function, only
  for polynomials possible
  \begin{description}
  \item[rt.phase\_fcn.apriori] scaling factor for the coefficients
    (1.0 means the values read from file.in.phase\_fcn are used as
    they are)
  \item[rt.phase\_fcn.sigma] covariance for the retrieval 
  \end{description}
\item[rt.phase] NOTE: only either this or rt.phase\_fcn is possible.
  \begin{itemize}
  \item determines the constant phase offset
  \end{itemize}
  \begin{description}
  \item[rt.phase.apriori] apriori scaling factor for the phase coefficient (1.0 means no change to the default)
  \item[rt.phase.sigma] covariance for the retrieval 
  \end{description}
  \item 
\end{description}

The parameters for the emipical apodisationa and phase are given in
the files file.in.modulation\_fcn and file.in.phase\_fcn. The
structure of those files is described in the next sections. 




\subsection{file.in.modulation\_fcn}

The
general structure is 
\begin{description}
\item[fw.apod\_fcn = 1]\hspace{2cm}
  
  \noindent
  \begin{tabular}{ll}  
    jeaps & number of values (columns) in this file\\
    eapf(1) eapf(2) ... eapf(jeaps)& at OPD given in row below\\
    eapx(1) eapx(2) ... eapx(jeaps) & OPD\\
  \end{tabular}
\item[fw.apod\_fcn = 2,3]\hspace{2cm}
  \noindent
  \begin{tabular}{ll}  
    jeaps & number of values (columns) in this file\\
    eapf(1) eapf(2) ... eapf(jeaps)& coefficients of polynom or Fourier series\\
  \end{tabular}

  The apodisation function $eapdz$ as polynome series is given by
\begin{equation}
  eapdz = 1 + (eapf(1)-1)*(x/xmax) + (eapf(2)-1)*(x/xmax)^2 + \ldots
\end{equation}

  The apodisation function $eapdz$ as Fourier series is given by
  \begin{eqnarray}
    eapdz &=& 1 + (1-eapf(2))*sin(2*pi*eapf(1)*x/xmax) +\\
          &&(1-eapf(3))*cos(2*pi*eapf(1)*x/xmax) +\\
          &&(1-eapf(4))*sin(4*pi*eapf(1)*x/xmax) +\\
          &&(1-eapf(5))*cos(4*pi*eapf(1)*x/xmax) +\ldots    
  \end{eqnarray}
\item[fw.apod\_fcn=4] outputfile modulat.dat created by linefit
\end{description}
The parameters are setup such that the term $n$ has no effect if
eapf(n) is set to 1. It has been done so that the parameter value reflects the epadz at maximum OPD.

\subsection{file.in.phase\_fcn}

The general structure is
\begin{description}
\item[fw.apod\_fcn = 1]\hspace{2cm}
  
  \noindent
  \begin{tabular}{ll}  
    jephs & number of values (columns) in this file\\
    ephsf(1) ephsf(2) ... ephsf(jephs)& at OPD given in row below\\
    ephsx(1) ephsx(2) ... ephsx(jephs) & OPD\\
  \end{tabular}
\item[fw.apod\_fcn = 2] NOTE: 3 does not exist
  
  \noindent
  \begin{tabular}{ll}  
    jephs & number of values (columns) in this file\\
    ephsf(1) ephsf(2) ... ephsf(jephs)& coefficients of polynom or Fourier series\\
  \end{tabular}
  The phase  function $ephsdz$ as polynome series is given by
\begin{equation}
  ephsdz = (ephsf(1) -1) + (ephs(2)-1)*(x/xmax) + (ephs(3)-1)*(x/xmax)^2 + \ldots
\end{equation}
  NOTE: jephs must be fw.phase\_fcn.order +1 because the first value
  is the 0th order term. The term $n$ has no effect if ephsf(n) us set to 1!
  
\item[fw.apod\_fcn=4] outputfile modulat.dat created by linefit
\end{description}




\end{document}
