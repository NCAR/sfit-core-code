\documentclass[a4paper]{article}

\begin{document}

\section{Empirical apodisation and phase function}


The empirical apodisation and phase deals with ILS errors due to
misalignemt of the instrument.

To switch it on set the following flags in the fw section:
\begin{description}
\item [fw.apod\_fcn] 
  fw.apod\_function = T\\
  fw.apod\_function.type = 0,1,2,3 or 4\\
  fw.apod\_function.order = 1,2,3,\ldots
\item [fw.phase\_fcn] 
  fw.phase\_function = T\\
  fw.phase\_function.type = 0,1,2 or 4 (3 does not exist)\\
  fw.phase\_function.order = 0, 1,2,3,\ldots\\
\item[rt.phase]
  \begin{itemize}
  \item set as apriori value in rt.phase.apriori
  \item determines the constant phase offset
  \item the function determines in fw.phase\_fcn is added to it
  \end{itemize}
\end{description}
 
The structure of the files file.in.modulation\_fcn is

\vspace{1cm}
jeaps specifies the number of entries being read in\\
eapf(1) eapf(2) ... eapf(jeap)\\
$[$eapx(1) eapx(2) ... eapx(jeap)$]$\\
\vspace{1cm}
eapx is read only when ieap=1.
\begin{itemize}
\item   jeap simply specifies the length of the following vectors. eapf is the
  vector of apodization parameters, whose meaning depends on ieap: 
\end{itemize}

The structure of the files file.in.phase\_fcn is

\vspace{1cm}
jephs specifies the number of entries being read in\\
\begin{description}
\item[fw.phase\_function.type = 2:] iephs(0) iephs(1) iephs(2) \ldots iephs(jeap-1)\\
  NOTE: jephs must be fw.phase\_fcn.order + 1
\item[[fw.phase\_function.type = 1:] two lines:
  iephs(1) iephs(2) ... iephs(jeap)
  eapx(1) eapx(2) ... eapx(jeap)\\
\end{description}

\begin{description}
\item [fw.apod\_fcn.type=1:] eapf contains the values of the apodization function eapdz; 
  in this case eapx contains the values of path difference at 
  which the function is specified.
\item[fw.apod\_fcn.type=2:] eapf contains the coefficients of a polynomial, defined
  by
\end{description}
\begin{equation}
  eapdz = 1 + (eapf(1)-1)*(x/xmax) + (eapf(2)-1)*(x/xmax)^2 + \ldots
\end{equation}
where $x$ is the path difference
\begin{description}
\item [fw.apod\_fcn.type=3:] eapf contains the coefficients of a Fourier series
\end{description}
\begin{eqnarray}
  eapdz &=& 1 + (1-eapf(2))*sin(2*pi*eapf(1)*x/xmax) +\\
        &&(1-eapf(3))*cos(2*pi*eapf(1)*x/xmax) +\\
        &&(1-eapf(4))*sin(4*pi*eapf(1)*x/xmax) +\\
        &&(1-eapf(5))*cos(4*pi*eapf(1)*x/xmax) +\ldots    
\end{eqnarray}
\begin{description}
\item [fw.apod\_fcn.type=4:] file.in.modulation\_fcn is a file which contains the output of linefit (modulat.dat)
\end{description}


\begin{description}
\item [fw.phase\_fcn.type=1:] file.in.phase\_fcn is a file which contains the values of the phase function at specified optical path differencesa 
\item[fw.phase\_fcn.type=2:]  file.in.phase\_fcn contains the coefficients of a polynomial, defined
  by
\end{description}
\begin{equation}
  eapdz = ephs(0) + (eapf(1)-1)*(x/xmax) + (eapf(2)-1)*(x/xmax)^2 + \ldots
\end{equation}
where $x$ is the normalized path difference
\begin{description}
\item [fw.phase\_fcn.type=3:] does not exist.
\item [fw.phase\_fcn.type=4:] file.in.phase\_fcn is a file which contains the output of linefit (modulat.dat)
\end{description}

\end{document}
