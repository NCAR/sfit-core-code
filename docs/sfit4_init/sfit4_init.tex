% This file was converted to LaTeX by Writer2LaTeX ver. 1.4
% see http://writer2latex.sourceforge.net for more info
\documentclass{article}
\usepackage[ascii]{inputenc}
\usepackage[T1]{fontenc}
\usepackage[ngerman,english]{babel}
\usepackage{amsmath}
\usepackage{amssymb,amsfonts,textcomp}
\usepackage{color}
\usepackage{array}
\usepackage{supertabular}
\usepackage{hhline}
\usepackage{hyperref}
\hypersetup{colorlinks=true, linkcolor=blue, citecolor=blue, filecolor=blue, urlcolor=blue, pdftitle=, pdfauthor=Mathias Palm, pdfsubject=, pdfkeywords=}
%\newcommand\textsubscript[1]{\ensuremath{{}_{\text{#1}}}}
% Outline numbering
\setcounter{secnumdepth}{0}
\makeatletter
\newcommand\arraybslash{\let\\\@arraycr}
\makeatother
% List styles
\newcounter{saveenum}
\newcommand\liststyleWWviiiNumii{%
\renewcommand\labelitemi{{\textbullet}}
\renewcommand\labelitemii{${\circ}$}
\renewcommand\labelitemiii{${\blacksquare}$}
\renewcommand\labelitemiv{{\textbullet}}
}
\newcommand\liststyleWWviiiNumiii{%
\renewcommand\theenumi{\arabic{enumi}}
\renewcommand\theenumii{\arabic{enumii}}
\renewcommand\theenumiii{\arabic{enumiii}}
\renewcommand\theenumiv{\arabic{enumiv}}
\renewcommand\labelenumi{\theenumi.}
\renewcommand\labelenumii{\theenumii.}
\renewcommand\labelenumiii{\theenumiii.}
\renewcommand\labelenumiv{\theenumiv.}
}
% Page layout (geometry)
\setlength\paperwidth{16.5354in}
\setlength\paperheight{11.6929in}
\setlength\voffset{-1in}
\setlength\hoffset{-1in}
\setlength\topmargin{1in}
\setlength\oddsidemargin{1in}
\setlength\textheight{9.6929in}
\setlength\textwidth{14.5354in}
\setlength\footskip{0.0cm}
\setlength\headheight{0cm}
\setlength\headsep{0cm}
% Footnote rule
\setlength{\skip\footins}{0.0469in}
\renewcommand\footnoterule{\vspace*{-0.0071in}\setlength\leftskip{0pt}\setlength\rightskip{0pt plus 1fil}\noindent\textcolor{black}{\rule{0.25\columnwidth}{0.0071in}}\vspace*{0.0398in}}
% Pages styles
\makeatletter
\newcommand\ps@Standard{
  \renewcommand\@oddhead{}
  \renewcommand\@evenhead{}
  \renewcommand\@oddfoot{}
  \renewcommand\@evenfoot{}
  \renewcommand\thepage{\arabic{page}}
}
\makeatother
\pagestyle{Standard}
\setlength\tabcolsep{1mm}
\renewcommand\arraystretch{1.3}
\title{}
\author{Mathias Palm}
\date{2019-11-14}
\begin{document}
\clearpage\setcounter{page}{1}\pagestyle{Standard}
{\centering\bfseries
Initialisation of sfit4
\par}

\section{}
{
The general format is }

{
Keyword1.keyword2.keyword3 \ = \ value}


\bigskip

\section{Keyword 1}

Defines general section under which keyword2,3,4,{\dots} are defining more detailed parameters

\begin{description}
\item[file] defines in and output files. All output files are also set to default values by the code.
\item[gas] defines parameters for the retrieval gases
\item[fw] all parameters for the forward model are defined in this section
\item[kb] is set to true if Kb matrizes are calculated at the end of the talk.
\item[rt] in this section the retrieval parameters are defined, except for the $S_A$ matrices for the gases which are
defined in the gas section. If rt is set to F, no retrieval is performed but only a forward calculation.
\item[band] the parameters for all MW bands are defined. 
\item[sp] contains additional noise information for the spectra, i.e. for deweighting
\item[out] more details of output are defined
\end{description}

\noindent
{\textbf{Do not rely on default values for in and output to make the program forward compatible}

\begin{flushleft}
\tablefirsthead{}
\tablehead{}
\tabletail{}
\tablelasttail{}
\begin{supertabular}{|m{2.10666in}|m{1.7969599in}|m{0.6087598in}m{0.54345983in}|m{5.30046in}|m{0.74065983in}|}
\hline
{\bfseries Key} &
{\bfseries Dependency} &
\multicolumn{2}{m{1.2309599in}|}{{\bfseries Default}} &
{\bfseries Description} &
{\bfseries versions}\\\hline
{\ttfamily file.in.spectrum } &
~
 &
\multicolumn{2}{m{1.2309599in}|}{{\ttfamily t15asc.4}} &
{\ttfamily File containing the spectrum in ASCII } &
~
\\\hline
{\ttfamily file.in.stalayers} &
~
 &
\multicolumn{2}{m{1.2309599in}|}{{\ttfamily station.layers}} &
{\ttfamily File containing the layering} &
~
\\\hline
{\ttfamily File.in.refprofile} &
~
 &
\multicolumn{2}{m{1.2309599in}|}{{\ttfamily reference.prf}} &
{\ttfamily File containing the atmosphere } &
~
\\\hline
{\ttfamily file.in.modulation\_fcn} &
{\ttfamily fw.apod\_fcn = T} &
\multicolumn{2}{m{1.2309599in}|}{{\ttfamily modulation\_gcn.dat}} &
{\ttfamily empirical apodization function}

{\ttfamily fw.apod\_fcn.type=: table containing the apodisation versus the OPD}

{\ttfamily fw.apod\_fcn.type = 2,3: coefficients of the polynomial or Fourier series.}

{\ttfamily fw.apod\_fcn.type = 4: Output of lft (modulat.dat)} &
~
\\\hline
{\ttfamily file.in.phase\_fcn} &
{\ttfamily fw.phase\_fcn = T} &
\multicolumn{2}{m{1.2309599in}|}{{\ttfamily Phase.dat}} &
{\ttfamily empirical phase function}

{\ttfamily fw.phase\_fcn.type=1: table of values version OPD}

{\ttfamily fw.phase\_fcn.type = 2: coefficients of the polynomial, Must be one more than
ft.phase\_fcn.order. }

{ \texttt{The first coefficient is the 0}\texttt{\textsuperscript{th}}\texttt{ order (constant
over the whole interferogram)}}

{\ttfamily fw.phase\_fcn.type = 4: Output of lft (modulat.dat)} &
~

~

~

~

{\ttfamily {\textgreater}= 1.0}\\\hline
{\ttfamily file.in.sa\_matrix} &
{\ttfamily gas.x.correlation = T}

{\ttfamily gas.x.correlation.type = 4} &
\multicolumn{2}{m{1.2309599in}|}{{\ttfamily sainv.input}} &
{\ttfamily file containing a full sa matrix } &
\\\hline
&
{\ttfamily gas.x.correlation = T gas.x.correlation.type = 5} &
\multicolumn{2}{m{1.2309599in}|}{{\ttfamily sainv.input}} &
{\ttfamily file containing fill inverse matrix $S_a^{-1}$} &
\\\hline
{\ttfamily file.in.isotope} &
{\ttfamily fw.isotope\_separation = T} &
\multicolumn{2}{m{1.2309599in}|}{{\ttfamily isotope.input}} &
{\ttfamily containg the isotope description} &
~
\\\hline
{\ttfamily file.in.solarlines} &
{\ttfamily fw.solspectrum = T} &
\multicolumn{2}{m{1.2309599in}|}{{\ttfamily solar\_data.input}} &
{\ttfamily containg solar lines} &
~
\\\hline
{\ttfamily file.in.linelist} &
~
 &
\multicolumn{2}{m{1.2309599in}|}{{\ttfamily none}} &
{\ttfamily file containing spectral data, created with hbin} &
~
\\\hline
{\ttfamily file.in.transmission} &
{\ttfamily fw.filter\_transmission = T} &
\multicolumn{2}{m{1.2309599in}|}{{\ttfamily none}} &
{\ttfamily A file containing a measured transmission. The format is the same as
file.in.spectrum} &
~
\\\hline
~
 &
~
 &
\multicolumn{2}{m{1.2309599in}|}{~
} &
~
 &
~
\\\hline
{\ttfamily file.out.solarspectrum} &
{\ttfamily fw.solar\_spectrum=T} &
\multicolumn{2}{m{1.2309599in}|}{{\ttfamily solspec.dat}} &
{\ttfamily Calculated solar spectrum} &
~
\\\hline
{\ttfamily file.out.summary} &
~
 &
\multicolumn{2}{m{1.2309599in}|}{{\ttfamily summary}} &
~
 &
~
\\\hline
{\ttfamily file.out.pbpfile} &
~
 &
\multicolumn{2}{m{1.2309599in}|}{{\ttfamily pbpfile}} &
~
 &
~
\\\hline
{\ttfamily file.out.statevec} &
~
 &
\multicolumn{2}{m{1.2309599in}|}{{\ttfamily statevec}} &
~
 &
~
\\\hline
{\ttfamily file.out.k\_matrix} &
~
 &
\multicolumn{2}{m{1.2309599in}|}{{\ttfamily k.out}} &
~
 &
~
\\\hline
{\ttfamily file.out.g\_matrix} &
~
 &
\multicolumn{2}{m{1.2309599in}|}{{\ttfamily d.complete}} &
~
 &
~
\\\hline
{\ttfamily file.out.shat\_matrix} &
~
 &
\multicolumn{2}{m{1.2309599in}|}{{\ttfamily shat.complete}} &
~
 &
~
\\\hline
{\ttfamily file.out.sa\_matrix} &
~
 &
\multicolumn{2}{m{1.2309599in}|}{{\ttfamily sa.complete}} &
~
 &
~
\\\hline
{\ttfamily file.out.retprofiles} &
~
 &
\multicolumn{2}{m{1.2309599in}|}{{\ttfamily rprfs.table}} &
~
 &
~
\\\hline
{\ttfamily file.out.aprprofiles} &
~
 &
\multicolumn{2}{m{1.2309599in}|}{{\ttfamily aprfs.table}} &
~
 &
~
\\\hline
{\ttfamily file.out.ak\_matrix} &
~
 &
\multicolumn{2}{m{1.2309599in}|}{{\ttfamily ak.out}} &
~
 &
~
\\\hline
{\ttfamily file.out.ab\_matrix} &
~
 &
\multicolumn{2}{m{1.2309599in}|}{{\ttfamily ab.out}} &
~
 &
~
\\\hline
{\ttfamily file.out.parm\_vectors} &
~
 &
\multicolumn{2}{m{1.2309599in}|}{{\ttfamily parm.vectors}} &
{\ttfamily Internal statevector per iteration} &
~
\\\hline
{\ttfamily file.out.seinv\_vector} &
~
 &
\multicolumn{2}{m{1.2309599in}|}{{\ttfamily seinv.out}} &
~
 &
~
\\\hline
{\ttfamily file.out.sainv\_matrix} &
~
 &
\multicolumn{2}{m{1.2309599in}|}{{\ttfamily sainv.out}} &
~
 &
~
\\\hline
{\ttfamily file.out.smeas\_matrix} &
~
 &
\multicolumn{2}{m{1.2309599in}|}{{\ttfamily smeas.target}} &
~
 &
~
\\\hline
{\ttfamily file.out.ssmooth\_matrix} &
~
 &
\multicolumn{2}{m{1.2309599in}|}{{\ttfamily ssmooth.target}} &
~
 &
~
\\\hline
{\ttfamily file.out.kb\_matrix} &
~
 &
\multicolumn{2}{m{1.2309599in}|}{{\ttfamily kb.out}} &
~
 &
~
\\\hline
\multicolumn{5}{|m{10.67126in}|}{~
} &
~
\\\hline
{\ttfamily gas.layers} &
~
 &
\multicolumn{2}{m{1.2309599in}|}{~
} &
{\ttfamily Nr of layers the gas is retrieved on. Must match the number of layers in
file.statlayers} &
~
\\\hline
{\ttfamily gas.profile.list} &
~
 &
\multicolumn{2}{m{1.2309599in}|}{{\ttfamily empty}} &
{\ttfamily names of the gases for which profiles are retrieved } &
~
\\\hline
{\ttfamily gas.column.list} &
~
 &
\multicolumn{2}{m{1.2309599in}|}{{\ttfamily empty}} &
{\ttfamily names of the gases for which columns are retrieved} &
~
\\\hline
{\ttfamily gas.profile.x.correlation} &
~
 &
\multicolumn{2}{m{1.2309599in}|}{{\ttfamily F}} &
{\ttfamily T for calculation of off diagonal correlation} &
~
\\\hline
{\ttfamily gas.profile.x.correlation.type} &
{\ttfamily gas.profile.x.correlation = T} & \multicolumn{2}{m{1.2309599in}|}{~} &
{ \texttt{definition of off diagonal correlation in the
S}\texttt{\textsubscript{a}}\texttt{{}-matrix}}

{\ttfamily 1 -- Gaussian shape}

{\ttfamily 2 -- Exponential shape }

{ \texttt{4 - the S}\texttt{\textsubscript{a}}\texttt{ matrix is read in from file.sa\_matrix}}

{ \texttt{5 -- the inverse, S}\texttt{\textsubscript{a}}\texttt{\textsuperscript{{}-1}}\texttt{,
matrix is read from file.sa\_matrix}}

{\ttfamily 6 -- an L1 redularization matrix is created} &
~

~

~

~

~

{\ttfamily {\textgreater}=1.0}\\\hline
{\ttfamily gas.profile.x.correlation.width} &
{ \texttt{gas.profile.x.correlation = T}}

{\ttfamily gas.profile.x.correlation.type =1,2} &
\multicolumn{2}{m{1.2309599in}|}{~
} &
{\ttfamily width of the correlation in km} &
~
\\\hline
{\ttfamily gas.profile.x.correlation.minalt} &
{\ttfamily gas.profile.x.correlation = T}

{\ttfamily gas.profile.x.correlation.type =1,2} &
\multicolumn{2}{m{1.2309599in}|}{~
} &
{\ttfamily correlation calculation starts at given altitude} &
~
\\\hline
{\ttfamily gas.profile.x.correlation.maxalt} &
{\ttfamily gas.profile.x.correlation = T}

{\ttfamily gas.profile.x.correlation.type =1,2} &
\multicolumn{2}{m{1.2309599in}|}{~
} &
{\ttfamily correlation calculation ends at given altitude} &
~
\\\hline
{\ttfamily Gas.profile.x.correlation.lambda} &
{\ttfamily gas.profile.x.correlation = T}

{ \foreignlanguage{ngerman}{\texttt{gas.profile.x.correlation.type = 6}}} &
\multicolumn{2}{m{1.2309599in}|}{~
} &
{\ttfamily Regularistion parameter for L1 regularisation} &
~
\\\hline
{\ttfamily gas.profile.x.logstate} &
~
 &
\multicolumn{2}{m{1.2309599in}|}{{\ttfamily F}} &
{\ttfamily If T the statevector is ln(VMR)} &
~
\\\hline
{\ttfamily gas.profile.x.scale} &
~
 &
\multicolumn{2}{m{1.2309599in}|}{~
} &
{\ttfamily apriori scaling of the VMR} &
~
\\\hline
{\ttfamily gas.profile.x.sigma} &
{\ttfamily gas.x.ifoff=0,1,2} &
\multicolumn{2}{m{1.2309599in}|}{~
} &
{\ttfamily diagonals of Sa matrix in fractions of the a priori }

{\ttfamily Nr of entries must correspond to the number of }

{\ttfamily layers defined in the statlayers} &
~
\\\hline
 &
{\ttfamily gas.x.ifoff=5} &
\multicolumn{2}{m{1.2309599in}|}{~
} &
{\ttfamily the corresponding rows of the read matrix are multiplied by the inverse value of the
sigma} &
~
\\\hhline{~-----}
~
 &
~
 &
\multicolumn{2}{m{1.2309599in}|}{~
} &
~
 &
~
\\\hline
{\ttfamily fw.tips} &
~
 &
\multicolumn{2}{m{1.2309599in}|}{{\ttfamily T}} &
{\ttfamily Turn TIPS On or Off. \ TIPS (as of HITRAN 2016)is slow but more accurate calculation
of partition sums.} &
~
\\\hline
{\ttfamily fw.isotope\_separation} &
~
 &
\multicolumn{2}{m{1.2309599in}|}{{\ttfamily F}} &
{\ttfamily Isotopes are separated, see file {\textless}file.in.isotope{\textgreater} for a
definition of the isotope separation } &
~
\\\hline
{\ttfamily fw.delnu} &
~
 &
\multicolumn{2}{m{1.2309599in}|}{~
} &
{\ttfamily Half width of integration interval for cross section calculation} &
~
\\\hline
{\ttfamily fw.lshapemodel} &
~
 &
\multicolumn{2}{m{1.2309599in}|}{{\ttfamily 0}} &
{\ttfamily Lineshape model }

{\ttfamily 0 -- depends on the spectroscopic values given \ }

{\ttfamily 1 - always Voigt}

{\ttfamily 2 - Galatry if BETA\_T is given, if not Voigt}

{\ttfamily 3 -- LM calculation using the Voigt profile by Boone (2012)}

{\ttfamily 4 -- Use the pCqSDHC lineshape model Tran(2013)} &
~
\\\hline
{\ttfamily fw.lshapemodel.sdv} &
{\ttfamily fw.lshapemodel = 4} &
\multicolumn{2}{m{1.2309599in}|}{{\ttfamily F}} &
{\ttfamily Use the SDV approximation in the pCqSDHC model.} &
~
\\\hline
{\ttfamily fw.linemixing} &
{\ttfamily fw.lshapemodel = 3,4} &
\multicolumn{2}{m{1.2309599in}|}{{\ttfamily F}} &
{\ttfamily if T and parameters found, linemixing is included}

{ \texttt{currently only 1}\texttt{\textsuperscript{st}}\texttt{ order pproximation (Rosenkranz,
1975)}} &
~
\\\hline
{\ttfamily fw.linemixing.gas} &
~
 &
\multicolumn{2}{m{1.2309599in}|}{~
} &
{\ttfamily gas for which linemixing is calculated } &
~
\\\hline
{\ttfamily fw.solar\_spectrum} &
~
 &
\multicolumn{2}{m{1.2309599in}|}{{\ttfamily F}} &
{\ttfamily if T inclusion of solar lines (files.solarlines)} &
~
\\\hline
{\ttfamily fw.pressure\_shift} &
~
 &
\multicolumn{2}{m{1.2309599in}|}{{\ttfamily F}} &
{\ttfamily Pressure induced line shift }

{\ttfamily T - read from linelist}

{\ttfamily F - no shift} &
~
\\\hline
{\ttfamily fw.apod\_fcn} &
~
 &
\multicolumn{2}{m{1.2309599in}|}{{\ttfamily F}} &
{\ttfamily Calculate apodization function} &
~
\\\hline
{\ttfamily fw.apod\_fcn.type} &
{\ttfamily fw.apod\_fcn = T} &
\multicolumn{2}{m{1.2309599in}|}{~
} &
{\ttfamily Empirical apodiziation}

{\ttfamily 0 - no empirical apodiziation}

{\ttfamily 1 - tabular function is read in}

{\ttfamily 2 - polynomial }

{\ttfamily 3 - fourier series }

{\ttfamily 4 - linefit output is read in}

{\ttfamily An extra file in file.in.apod\_fcn has to be suplied if type
{\textless}{\textgreater} 0. Format of this file depends on the type.} &
~
\\\hline
{\ttfamily fw.apod\_fcn.order} &
{\ttfamily fw.apod\_fcn.type = 2 o. 3} &
\multicolumn{2}{m{1.2309599in}|}{~
} &
{\ttfamily Order of polynomial/fourier series } &
~
\\\hline
{\ttfamily fw.phase\_fcn} &
~
 &
\multicolumn{2}{m{1.2309599in}|}{{\ttfamily F}} &
{\ttfamily T if emirical phase is calculated} &
~
\\\hline
{\ttfamily fw.phase\_fcn.type} &
{\ttfamily fw.phase\_fcn = T } &
\multicolumn{2}{m{1.2309599in}|}{~
} &
{\ttfamily Empirical phase error}

{0 - no empirical phase}

{\ttfamily 1 - tabular function is read in}

{\ttfamily 2 - polynomial}

{\ttfamily 4 - linefit output}

{\ttfamily An extra file in file.in.phase\_fcn has to be suplied if type
{\textless}{\textgreater} 0. Format of this file depends on the type. (compare documentation sfit4\_EAPOD\_EPHASE} &
~
\\\hline
{\ttfamily fw.phase\_fcn.order} &
{\ttfamily fw.phase\_fcn = T }

{\ttfamily fw.phase\_fcn.type = 2} &
\multicolumn{2}{m{1.2309599in}|}{~
} &
{\ttfamily Order of polynomial} &
~
\\\hline
{\ttfamily fw.emission} &
~
 &
\multicolumn{2}{m{1.2309599in}|}{{\ttfamily F}} &
{\ttfamily if T emitted radiation from the atmosphere is calculated} &
~
\\\hline
{\ttfamily fw.emission.T\_infinity} &
~
 &
\multicolumn{2}{m{1.2309599in}|}{{\ttfamily n}} &
{\ttfamily Temperature (in K) of the }

{\ttfamily radiating object outside the atmosphere}

{\ttfamily Moon = 370.0}

{\ttfamily Sun \ = 6000.0}

{\ttfamily None = 2.7} &
~
\\\hline
{\ttfamily fw.emission.object} &
{\ttfamily fw.emission = T} &
\multicolumn{2}{m{1.2309599in}|}{~
} &
{\ttfamily Reflexion of solar light off object}

{\ttfamily .e. only emission is calculated, no reflection}

{\ttfamily .m. reflection of solar light of the moon (pre-alpha)} &
~
\\\hline
{\ttfamily fw.emission.normalized} &
{\ttfamily fw.emission = T} &
\multicolumn{2}{m{1.2309599in}|}{~
} &
{\ttfamily spectra are normalized to one (T) or not normalized \ (F)} &
~
\\\hline
{\ttfamily fw.raytonly} &
~
 &
\multicolumn{2}{m{1.2309599in}|}{{\ttfamily F}} &
{\ttfamily if T only calculate raytracing} &
~
\\\hline
{\ttfamily fw.filter\_transmission} &
~
 &
\multicolumn{2}{m{1.2309599in}|}{{\ttfamily F}} &
{\ttfamily If T applies the measured transmission contained in file.in.transmission to the
calculated spectrum} &
~
\\\hline
\multicolumn{5}{|m{10.67126in}|}{~
} &
~
\\\hline
{\ttfamily Cell} &
~
 &
\multicolumn{2}{m{1.2309599in}|}{{\ttfamily x}} &
{\ttfamily Number (placeholders) for cell} &
~
\\\hline
{\ttfamily Cell.x.temperature} &
~
 &
\multicolumn{2}{m{1.2309599in}|}{~
} &
{\ttfamily Temperature of CELL in K} &
~
\\\hline
{\ttfamily Cell.x.pressure} &
~
 &
\multicolumn{2}{m{1.2309599in}|}{~
} &
{\ttfamily Pressure in CELL in hPa. This is the total pressure, not the partial pressure of the
gas.} &
~
\\\hline
{\ttfamily Cell.x.gas} &
~
 &
\multicolumn{2}{m{1.2309599in}|}{~
} &
{\ttfamily GAS in CELL. For more than one GAS, a seperate CELL has to be defined. } &
~
\\\hline
{\ttfamily Cell.x.vmr} &
~
 &
\multicolumn{2}{m{1.2309599in}|}{~
} &
{\ttfamily VMR of the gas in CELL} &
~
\\\hline
{\ttfamily Cell.x.path} &
~
 &
\multicolumn{2}{m{1.2309599in}|}{~
} &
{\ttfamily Path of the cell in cm} &
~
\\\hline
\multicolumn{5}{|m{10.67126in}|}{~
} &
~
\\\hline
{\ttfamily kb} &
~
 &
\multicolumn{2}{m{1.2309599in}|}{{\ttfamily F}} &
{\ttfamily T if Kb matrix entries are calculated, if the respective statevector entries are not
retrieved, i.e. given kb.slope = T the Kb row for the slopes are only calculated it slope is not retrieved.} &
~
\\\hline
{\ttfamily kb.profile} &
~
 &
\multicolumn{2}{m{1.2309599in}|}{{\ttfamily F}} &
{\ttfamily Calculates AB matrix for a wrong assumed profile if the retrieved gas is a column} &
~
\\\hline
{\ttfamily kb.profile.gas} &
{\ttfamily kb.profile} &
\multicolumn{2}{m{1.2309599in}|}{~
} &
{\ttfamily For which gas an error for the retrieved profile is calculated?} &
~
\\\hline
{\ttfamily kb.sza} &
~
 &
\multicolumn{2}{m{1.2309599in}|}{{\ttfamily F}} &
{\ttfamily T if Kb calculation of the SZA} &
~
\\\hline
{\ttfamily kb.line} &
~
 &
\multicolumn{2}{m{1.2309599in}|}{{\ttfamily F}} &
{\ttfamily T if Kb calculation for line intensities} &
~
\\\hline
{\ttfamily kb.line.gas} &
~
 &
\multicolumn{2}{m{1.2309599in}|}{~
} &
{\ttfamily for which gases line parameters are calculated, default: all gases which are
retrieved}

~

{\ttfamily predefined values:}

{\ttfamily target -- calculation only for the target gas}

{\ttfamily retrieval -- kb are calculated for each gas which is retrieved.}

{\ttfamily individual gasnames are also possible} &
~
\\\hline
{\ttfamily kb.line.type} &
~
 &
\multicolumn{2}{m{1.2309599in}|}{~
} &
{\ttfamily 1 if all line parameters of a gas are perturbed together with the same perturbation
(this is the only parameter supported so far)}

~

{\ttfamily Kb are calculated for:}

\liststyleWWviiiNumii
\begin{itemize}
\item {\ttfamily Intensity}
\item {\ttfamily Pressure broadening}
\item {\ttfamily Temperature dependency of pressure broadening}
\end{itemize}
 &
~
\\\hline
{\ttfamily kb.temperature} &
~
 &
\multicolumn{2}{m{1.2309599in}|}{~
} &
~
 &
~
\\\hline
{\ttfamily kb.slope} &
~
 &
\multicolumn{2}{m{1.2309599in}|}{~
} &
~
 &
~
\\\hline
{\ttfamily kb.curvature} &
~
 &
\multicolumn{2}{m{1.2309599in}|}{~
} &
~
 &
~
\\\hline
{\ttfamily kb.zshift} &
~
 &
\multicolumn{2}{m{1.2309599in}|}{~
} &
~
 &
~
\\\hline
{\ttfamily kb.omega} &
~
 &
\multicolumn{2}{m{1.2309599in}|}{~
} &
~
 &
~
\\\hline
{\ttfamily kb.max\_opd} &
~
 &
\multicolumn{2}{m{1.2309599in}|}{~
} &
~
 &
~
\\\hline
{\ttfamily kb.solstrnth} &
~
 &
\multicolumn{2}{m{1.2309599in}|}{~
} &
~
 &
~
\\\hline
{\ttfamily kb.solshft} &
~
 &
\multicolumn{2}{m{1.2309599in}|}{~
} &
~
 &
~
\\\hline
{\ttfamily kb.phase} &
~
 &
\multicolumn{2}{m{1.2309599in}|}{~
} &
~
 &
~
\\\hline
{\ttfamily kb.apod\_fnc} &
~
 &
\multicolumn{2}{m{1.2309599in}|}{~
} &
{\ttfamily if fw.apod\_fcn = F a three order three order polynomial is assumed as the empirical
apodisation function. If the apodisation function is read in as a table, no kb matrix is calculated (yet)} &
~
\\\hline
{\ttfamily kb.phase\_fcn} &
~
 &
\multicolumn{2}{m{1.2309599in}|}{~
} &
{\ttfamily if fw.phase\_fcn = F a three order three order polynomial is assumed as the empirical
apodisation function. If the phase function is read in as a table, no kb matrix is calculated (yet).} &
~
\\\hline
{\ttfamily kb.wshift} &
~
 &
\multicolumn{2}{m{1.2309599in}|}{~
} &
{\ttfamily not recommended (usually retrieved)} &
~
\\\hline
{\ttfamily kb.dwshift} &
~
 &
\multicolumn{2}{m{1.2309599in}|}{~
} &
{\ttfamily not recommended} &
~
\\\hline
\multicolumn{5}{|m{10.67126in}|}{~
} &
~
\\\hline
{\ttfamily rt} &
~
 &
\multicolumn{2}{m{1.2309599in}|}{~
} &
{\ttfamily Switch on (T) or off (F) Retrieval, if F only a forward model}

{\ttfamily calculation is performed} &
~
\\\hline
{\ttfamily rt.lm} &
~
 &
\multicolumn{2}{m{1.2309599in}|}{{\ttfamily F}} &
{\ttfamily Switch on (T) or off (F) LM iteration scheme} &
~
\\\hline
{\ttfamily rf.lm.gamma\_start} &
{\ttfamily rt.lm = T} &
\multicolumn{2}{m{1.2309599in}|}{~
} &
{\ttfamily Start value for gamma} &
~
\\\hline
{\ttfamily rf.lm.gamma\_inc} &
{\ttfamily rt.lm = T} &
\multicolumn{2}{m{1.2309599in}|}{~
} &
{\ttfamily Increase gamma by value if step was succesful} &
~
\\\hline
{\ttfamily rf.lm.gamma\_dec} &
{\ttfamily rt.lm = T} &
\multicolumn{2}{m{1.2309599in}|}{~
} &
{\ttfamily Decrease gamma if step failed } &
~
\\\hline
{\ttfamily rt.convergence} &
~
 &
\multicolumn{2}{m{1.2309599in}|}{~
} &
{\ttfamily convergence is reached if change in cost function is smaller than value} &
~
\\\hline
{\ttfamily rt.tolerance} &
{\ttfamily rt.lm = F OR \ \ \ }

{\ttfamily rt.convergance not given} &
\multicolumn{2}{m{1.2309599in}|}{~
} &
{\ttfamily convergence cirterion used by sfit2. Convergence is reached if the proposed change in
the spectrum is smaller than the noise * rt.tolerance} &
~
\\\hline
{\ttfamily rt.max\_iteration} &
~
 &
\multicolumn{2}{m{1.2309599in}|}{~
} &
{\ttfamily maximal number of iterations} &
~
\\\hline
{\ttfamily rt.wshift} &
~
 &
\multicolumn{2}{m{1.2309599in}|}{{\ttfamily F}} &
{\ttfamily T if a wavenumver shift is retrieved.} &
~
\\\hline
{\ttfamily rt.wshift.type} &
{\ttfamily rt.wshift = T} &
\multicolumn{2}{m{1.2309599in}|}{~
} &
{\ttfamily 0 - no shift for any bandpass}

{\ttfamily 1 - single shift for each bandpass}

{\ttfamily 2 - independent shift for each bandpass}

{\ttfamily 3 - idependent shift for each fit} &
~
\\\hline
{\ttfamily rt.wshift.apriori} &
{\ttfamily rt.wshift = T} &
\multicolumn{2}{m{1.2309599in}|}{~
} &
{\ttfamily apriori of the additional scaling for all microwindows -1. Internally it is added to
band.x.wavfac, the scaling which is applied to microwindow x as apriori is }

~

{\ttfamily band.x.wavfac + rt.wshift.apriori}

~

{\ttfamily after retrieval the value is found in statevec as IWNUMSHFT\_X}

~

{\ttfamily hence the scaling applied in a MW is}

~

{\ttfamily band.x.wavfac + IWNumShft\_x} &
~
\\\hline
{\ttfamily rt.wshift.sigma} &
{\ttfamily rt.wshift = T} &
\multicolumn{2}{m{1.2309599in}|}{~
} &
{\ttfamily its sa} &
~
\\\hline
{\ttfamily rt.slope} &
~
 &
\multicolumn{2}{m{1.2309599in}|}{{\ttfamily F}} &
{\ttfamily slope is retrieved if T} &
~
\\\hline
{\ttfamily rt.slope.apriori} &
{\ttfamily rt.slope = T} &
\multicolumn{2}{m{1.2309599in}|}{~
} &
{\ttfamily a priori of slope } &
~
\\\hline
{\ttfamily rt.slope.sigma} &
{\ttfamily rt.slope = T} &
\multicolumn{2}{m{1.2309599in}|}{~
} &
{\ttfamily sa of slope} &
~
\\\hline
{\ttfamily rt.curvature} &
~
 &
\multicolumn{2}{m{1.2309599in}|}{{\ttfamily F}} &
{\ttfamily curvature on spectrum is retrieved if T} &
~
\\\hline
{\ttfamily rt.curvature.apriori} &
{\ttfamily rt.curvature = T} &
\multicolumn{2}{m{1.2309599in}|}{~
} &
{\ttfamily a priori of curvature} &
~
\\\hline
{\ttfamily rt.curvature.sigma} &
{\ttfamily rt.curvature = T} &
\multicolumn{2}{m{1.2309599in}|}{~
} &
{\ttfamily sa of curvature} &
~
\\\hline
{\ttfamily rt.phase } &
~
 &
\multicolumn{2}{m{1.2309599in}|}{{\ttfamily F}} &
{\ttfamily simple phase correction retrieved if T} &
~
\\\hline
{\ttfamily rt.phase.apriori} &
{\ttfamily rt.phase = T} &
\multicolumn{2}{m{1.2309599in}|}{~
} &
~
 &
~
\\\hline
{\ttfamily rt.phase.sigma} &
{\ttfamily rt.phase = T} &
\multicolumn{2}{m{1.2309599in}|}{~
} &
~
 &
~
\\\hline
{\ttfamily rt.phase\_fcn} &
{\ttfamily fw.phase\_fcn = T}

{\ttfamily fw.phase\_fcn.type=2} &
\multicolumn{2}{m{1.2309599in}|}{{\ttfamily F}} &
{ Empirical phase function retrieved if T} &
~
\\\hline
{\ttfamily rt.phase\_fcn.apriori} &
{\ttfamily rt.phase\_fcn = T} &
\multicolumn{2}{m{1.2309599in}|}{~
} &
~
 &
~
\\\hline
{\ttfamily rt.phase\_fcn.phase} &
{\ttfamily rt.phase\_fcn = T} &
\multicolumn{2}{m{1.2309599in}|}{~
} &
~
 &
~
\\\hline
{\ttfamily rt.apod\_fcn} &
{\ttfamily fw.apod\_fcn = T}

{\ttfamily fw.apod\_fcn.type=2,3} &
\multicolumn{2}{m{1.2309599in}|}{{\ttfamily F}} &
{ Empirical phase function retrieved if T} &
~
\\\hline
{\ttfamily rt.apod\_fcn.apriori} &
{\ttfamily rt.apod\_fcn = T} &
\multicolumn{2}{m{1.2309599in}|}{~
} &
~
 &
~
\\\hline
{\ttfamily rt.apod\_fcn.phase} &
{\ttfamily rt.apod\_fcn = T} &
\multicolumn{2}{m{1.2309599in}|}{~
} &
~
 &
~
\\\hline
{\ttfamily rt.solshift} &
{\ttfamily fw.solar\_spectrum = T} &
\multicolumn{2}{m{1.2309599in}|}{{\ttfamily F}} &
{\ttfamily retrieve shift in solar lines if T} &
~
\\\hline
{\ttfamily rt.solshift.apriori} &
{\ttfamily rt.solshift = T} &
\multicolumn{2}{m{1.2309599in}|}{~
} &
~
 &
~
\\\hline
{\ttfamily rt.solshift.sigma} &
{\ttfamily rt.solshift = T} &
\multicolumn{2}{m{1.2309599in}|}{~
} &
~
 &
~
\\\hline
{\ttfamily rt.solstrnth} &
{\ttfamily fw.solar\_spectrum = T} &
\multicolumn{2}{m{1.2309599in}|}{{\ttfamily F}} &
{\ttfamily retrieve strength of solar lines if T} &
~
\\\hline
{\ttfamily rt.solstrnth.apriori} &
{\ttfamily rt.solshift = T} &
\multicolumn{2}{m{1.2309599in}|}{~
} &
~
 &
~
\\\hline
{\ttfamily rt.solstrnth.sigma} &
{\ttfamily rt.solshift = T} &
\multicolumn{2}{m{1.2309599in}|}{~
} &
~
 &
~
\\\hline
{\ttfamily rt.dwshift} &
~
 &
\multicolumn{2}{m{1.2309599in}|}{{\ttfamily F}} &
{\ttfamily if T retrieval of line shifts for each retrieved gas} &
~
\\\hline
~
 &
~
 &
\multicolumn{2}{m{1.2309599in}|}{~
} &
~
 &
~
\\\hline
{\ttfamily rt.temperature} &
~
 &
\multicolumn{2}{m{1.2309599in}|}{{\ttfamily F}} &
{\ttfamily if T, temperature is retrieved \ } &
~
\\\hline
{\ttfamily rt.temperature.sigma} &
{\ttfamily rt.temperature = T} &
\multicolumn{2}{m{1.2309599in}|}{~
} &
{\ttfamily diagonals of sa matrix for temperature for each layer in state}

{\ttfamily vector} &
~
\\\hline
\multicolumn{3}{|m{4.66986in}|}{~
} &
\multicolumn{2}{m{5.92266in}|}{~
} &
~
\\\hline
{\ttfamily band} &
~
 &
\multicolumn{2}{m{1.2309599in}|}{~
} &
{\ttfamily = 1..2 MWs that are included in the calculation} &
~
\\\hline
{\ttfamily band.x.nu\_start} &
~
 &
\multicolumn{2}{m{1.2309599in}|}{~
} &
{\ttfamily smallest frequency of MW} &
~
\\\hline
{\ttfamily band.x.nu\_stop} &
~
 &
\multicolumn{2}{m{1.2309599in}|}{~
} &
{\ttfamily argest frequency of MW} &
~
\\\hline
{\ttfamily band.x.calc\_point\_space} &
~
 &
\multicolumn{2}{m{1.2309599in}|}{~
} &
{\ttfamily spacing for spectrum calculation} &
~
\\\hline
{\ttfamily band.x.wave\_factor} &
~
 &
\multicolumn{2}{m{1.2309599in}|}{{\ttfamily 1.0}} &
{\ttfamily scaling of wave factor in this band} &
~
\\\hline
{\ttfamily band.x.opd\_max} &
~
 &
\multicolumn{2}{m{1.2309599in}|}{~
} &
{\ttfamily maximal OPD for this band} &
~
\\\hline
{\ttfamily band.x.omega} &
~
 &
\multicolumn{2}{m{1.2309599in}|}{~
} &
{\ttfamily FOV for this band} &
~
\\\hline
{\ttfamily band.x.apodization\_code} &
~
 &
\multicolumn{2}{m{1.2309599in}|}{{\ttfamily 0}} &
{\ttfamily Imposed Apodization Code }

{\ttfamily 0 -- Boxcar}

{\ttfamily 1 -3 Norton Beer (weak, med., strng)}

{\ttfamily 4 - Denver data}

{\ttfamily 5 -- Triangle}

{\ttfamily 6 - Happ -- Genzel}

{\ttfamily 7 -- KPNO Atmospheric Spectra A}

{\ttfamily 8 -- KPNO Atmospheric Spectra B}

{\ttfamily 9 -- Hamming function} &
~
\\\hline
{\ttfamily band.x.zshift} &
~
 &
\multicolumn{2}{m{1.2309599in}|}{{\ttfamily F}} &
{\ttfamily T if an offset is retrieved in this band} &
~
\\\hline
{\ttfamily band.x.zshift.type} &
{\ttfamily band.1.zshift = T} &
\multicolumn{2}{m{1.2309599in}|}{~
} &
{\ttfamily 0 - use the a priori as given}

{\ttfamily 1 - allow to retrieve for each bad}

{\ttfamily 2 - use zero level from first band in list} &
~
\\\hline
{\ttfamily band.x.zshift.apriori} &
{\ttfamily band.1.zshift = T} &
\multicolumn{2}{m{1.2309599in}|}{~
} &
{\ttfamily apriori of shift of the zero line} &
~
\\\hline
{\ttfamily band.x.zshift.sigma} &
{\ttfamily band.1.zshift = T}

{\ttfamily band.1.zshift.type = \{1,2\}} &
\multicolumn{2}{m{1.2309599in}|}{~
} &
{\ttfamily sa of the zero line shift} &
~
\\\hline
{\ttfamily band.x.beam} &
~
 &
\multicolumn{2}{m{1.2309599in}|}{{\ttfamily empty}} &
{\ttfamily Number of beams beams included}

{\ttfamily two lines for each beam} &
~
\\\hline
{\ttfamily band.x.beam.y.apriori} &
{\ttfamily band.1.beam /= 0} &
\multicolumn{2}{m{1.2309599in}|}{~
} &
{\ttfamily Four values:}

{\ttfamily amp, freq, phase, slope} &
~
\\\hline
{\ttfamily band.x.beam.y.sigma } &
{\ttfamily band.1.beam /= 0} &
\multicolumn{2}{m{1.2309599in}|}{~
} &
{\ttfamily Its standart deviation, if set to o.o not retrieved but fixed} &
~
\\\hline
{\ttfamily band.x.beam.model} &
~
 &
\multicolumn{2}{m{1.2309599in}|}{~
} &
{\ttfamily Channel model }

{\ttfamily PS - phase model}

{\ttfamily IP - interferogram pertubation model } &
~
\\\hline
{\ttfamily band.x.gasb} &
~
 &
\multicolumn{2}{m{1.2309599in}|}{~
} &
{\ttfamily gases which are retrieved from this band, must be contained in}

{\ttfamily gas (see above)} &
~
\\\hline
{\ttfamily band.x.tempretb} &
{\ttfamily rt.temperature = T} &
\multicolumn{2}{m{1.2309599in}|}{{\ttfamily F}} &
{\ttfamily T if temperature is retrieved in this band} &
~
\\\hline
{\ttfamily band.x.snr} &
~
 &
\multicolumn{2}{m{1.2309599in}|}{~
} &
{\ttfamily initial default snr for all scans in this band. \ Over ridden by snr from t15asc file
and sp window} &
~
\\\hline
\multicolumn{3}{|m{4.66986in}|}{~
} &
\multicolumn{2}{m{5.92266in}|}{~
} &
~
\\\hline
{\ttfamily sp.snr = x} &
~
 &
\multicolumn{2}{m{1.2309599in}|}{~
} &
{ which additional snr windows are taken into account}

{ e.g. = 1, the lines containing an 1 are read in, all other lines are ignored. example = 1 2 3}

{ this over rides all previous snr values in this window} &
~
\\\hline
{\ttfamily sp.snr.x.nu\_start} &
{\ttfamily spectrum.snr not empty} &
\multicolumn{2}{m{1.2309599in}|}{~
} &
{\ttfamily low wavenumber for snr window x} &
~
\\\hline
{\ttfamily sp.snr.x.nu\_stop} &
~
 &
\multicolumn{2}{m{1.2309599in}|}{~
} &
{\ttfamily high wavenumber for snr window x} &
~
\\\hline
{\ttfamily sp.snr.x.snr} &
~
 &
\multicolumn{2}{m{1.2309599in}|}{~
} &
{ snr value for window x} &
~
\\\hline
\multicolumn{3}{|m{4.66986in}|}{~
} &
\multicolumn{2}{m{5.92266in}|}{~
} &
~
\\\hline
{\ttfamily out.level} &
~
 &
\multicolumn{2}{m{1.2309599in}|}{{\ttfamily 0}} &
{\ttfamily Output level, a predefined set of putput files} &
~
\\\hline
{\ttfamily out.gas\_spectra} &
~
 &
\multicolumn{2}{m{1.2309599in}|}{{\ttfamily F}} &
{\ttfamily T for write gasfiles} &
~
\\\hline
{\ttfamily out.gas\_spectra.type} &
~
 &
\multicolumn{2}{m{1.2309599in}|}{~
} &
{\ttfamily Type of GASFILE}

{\ttfamily 1 - only the final spectrum, the spectrum}

{\ttfamily of each gas and the }

{\ttfamily solar spectrum will be printed out. }

{\ttfamily The files are named like}

{\ttfamily gas1.1.1, allgases.1.1 and solar 1.1}

{\ttfamily 2 - spectra will be printed out for each}

{\ttfamily iteration. The names are like above,but an}

{\ttfamily extra number is appended denoting the}

{\ttfamily iteration number the numbers appended to}

{\ttfamily the files are {\textquotedbl}nr of window{\textquotedbl}, {\textquotedbl}nr of
scan{\textquotedbl}}

{\ttfamily and {\textquotedbl}nr of iteration{\textquotedbl}}

{\ttfamily The information of for band nr, gas and}

{\ttfamily iteration number are also contained in the file}

{\ttfamily header} &
~
\\\hline
\multicolumn{5}{|m{10.67126in}|}{{\ttfamily Additionally to the predefined output acc to the
level given in {\textless}output{\textgreater} the following quantities can be written out. If such a key is given the
resp quantity is written out to the file defined by the parameter string (e.g. output.k-matrix = kk.out -- the Kmatrix
is written out to kk.out) }} &
~
\\\hline
{\ttfamily out.k\_matrix} &
~
 &
\multicolumn{2}{m{1.2309599in}|}{{\ttfamily F}} &
{\ttfamily {\textless}filename{\textgreater} matrices written in file \ (now only: K.out)} &
~
\\\hline
{\ttfamily Out.g\_matrix} &
~
 &
\multicolumn{2}{m{1.2309599in}|}{{\ttfamily F}} &
{\ttfamily Write out complete Gain matrix} &
~
\\\hline
{\ttfamily out.sa\_matrix } &
~
 &
\multicolumn{2}{m{1.2309599in}|}{{\ttfamily F}} &
{\ttfamily {\textless}filename{\textgreater} write out SA-matrix (now only: SA.out)} &
~
\\\hline
{\ttfamily out.smeas\_matrix} &
~
 &
\multicolumn{2}{m{1.2309599in}|}{{\ttfamily F}} &
{\ttfamily {\textless}filename{\textgreater} write out SM-matrix (error on profile due to the
measurement noise (now only: SM.out)} &
~
\\\hline
{\ttfamily out.shat\_matrix} &
~
 &
\multicolumn{2}{m{1.2309599in}|}{{\ttfamily F}} &
~
 &
~
\\\hline
{\ttfamily out.refprofiles} &
~
 &
\multicolumn{2}{m{1.2309599in}|}{{\ttfamily F}} &
~
 &
~
\\\hline
{\ttfamily out.aprprofiles} &
~
 &
\multicolumn{2}{m{1.2309599in}|}{{\ttfamily F}} &
~
 &
~
\\\hline
{\ttfamily out.ak\_matrix} &
~
 &
\multicolumn{2}{m{1.2309599in}|}{{\ttfamily F}} &
~
 &
~
\\\hline
{\ttfamily out.ab\_matrix} &
~
 &
\multicolumn{2}{m{1.2309599in}|}{{\ttfamily F}} &
~
 &
~
\\\hline
{\ttfamily out.summary} &
~
 &
\multicolumn{2}{m{1.2309599in}|}{{\ttfamily F}} &
~
 &
~
\\\hline
{\ttfamily out.pbpfile} &
~
 &
\multicolumn{2}{m{1.2309599in}|}{{\ttfamily F}} &
~
 &
~
\\\hline
{\ttfamily out.channel} &
~
 &
\multicolumn{2}{m{1.2309599in}|}{{\ttfamily F}} &
~
 &
~
\\\hline
{\ttfamily out.parm\_vectors} &
~
 &
\multicolumn{2}{m{1.2309599in}|}{{\ttfamily F}} &
~
 &
~
\\\hline
{\ttfamily out.ssmooth\_matrix} &
~
 &
\multicolumn{2}{m{1.2309599in}|}{{\ttfamily F}} &
~
 &
~
\\\hline
{\ttfamily out.sainv\_matrix} &
~
 &
\multicolumn{2}{m{1.2309599in}|}{{\ttfamily F}} &
~
 &
~
\\\hline
{\ttfamily out.seinv\_vector} &
~
 &
\multicolumn{2}{m{1.2309599in}|}{{\ttfamily F}} &
~
 &
~
\\\hline
{\ttfamily out.gas\_spectra} &
~
 &
\multicolumn{2}{m{1.2309599in}|}{{\ttfamily F}} &
{\ttfamily T calculated spectra for each retrieved gas and each band are printed out} &
~
\\\hline
{\ttfamily out.gas\_spectra.type} &
{\ttfamily output.gas\_spectra = T} &
\multicolumn{2}{m{1.2309599in}|}{{\ttfamily F}} &
\liststyleWWviiiNumiii
\begin{enumerate}
\item {\ttfamily gas spectra are only for the final iteration written out}
\item {\ttfamily gas spectra are written out for each iteration}
\end{enumerate}
 &
\liststyleWWviiiNumiii
\setcounter{saveenum}{\value{enumi}}
\begin{enumerate}
\setcounter{enumi}{\value{saveenum}}
\item ~

\end{enumerate}
\\\hline
{\ttfamily out.raytrace} &
~
 &
\multicolumn{2}{m{1.2309599in}|}{{\ttfamily F}} &
{\ttfamily Write out raytrace} &
~
\\\hline
{\ttfamily out.raytrace.type} &
~
 &
\multicolumn{2}{m{1.2309599in}|}{~
} &
{\ttfamily Type of raytracing output} &
~
\\\hline
\end{supertabular}
\end{flushleft}

\bigskip


\bigskip

{\bfseries
Output description}

{
Files which may appear but are not described here are a legacy and are subject to modification or removal in the future,
so dont relay on them, but notify the maintainers of sfit4 if you need the information contained in them.}


\bigskip

\begin{flushleft}
\tablefirsthead{}
\tablehead{}
\tabletail{}
\tablelasttail{}
\begin{supertabular}{|m{0.9629598in}|m{0.9108598in}|m{6.92336in}|}
\hline
{ OUTPUT} &
{ Contained in Output level} &
{ Description}\\\hline
{ PRFS.out} &
{ 0} &
{ contains all profiles of the retrieval gases together with the alitude grid, pressure ,
temperature and airmass (vertical). For each gas there are five columns:}

{ VMR Apriori}

{ VMR Retrieved}

{ SIGMA VMR RETRIEVED}

{ PARTIAL COLUMN A priori}

{ PARTIAL COLUMN Retrieved}\\\hline
{ pbpfile} &
{ 0} &
{ contains the retrieved, measured spectra and the difference thereof}\\\hline
{ AK.out} &
{ 0} &
{ Averaging kernel in units of the internal statevector, i.e. x/x\textsubscript{a}}\\\hline
{ Ab.out} &
{ 0} &
{ Contains the G\textsubscript{y}K\textsubscript{b} matrix (see formula 3.16 page 48 in Rodgers
(2000)}\\\hline
{ Kb.out} &
{ 1} &
{ contains the K\textsubscript{b} matrix for all parameters which are not retrieved (and
contained in the K-matrix)}\\\hline
{ SM.out} &
{ 1} &
{ contains the full matrix of the retrieval noise
}\\\hline
{ spc.*} &
{ 1} &
{ contains the spectra calculated for each retrieval gas, each iteration (if
output.gas\_spectra.type = 2) and each microwindow}\\\hline
\end{supertabular}
\end{flushleft}

\bigskip

{
References}

{
Boone, C. D.; Walker, K. A. \& Bernath, P. F. An efficient analytical approach for calculating line mixing in
atmospheric remote sensing applications J. Quant. Spectrosc. Radiat. Transfer, 2011, 112, 980 -- 989}

{
Rosenkranz PW. Shape of the 5 mm oxygen band in the atmosphere. IEEE Trans Antennas Propag 1975;AP-23:498--506.}

{
Tran, H.; Ngo, N. \& Hartmann, J.-M. Efficient computation of some speed-dependent isolated line profiles Journal of
Quantitative Spectroscopy and Radiative Transfer, 2013}
\end{document}
