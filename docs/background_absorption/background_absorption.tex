\documentclass{article}

\begin{document}
\section{Modeling of an absorbing layer}

An absorbing layer in the troposphere is of interest in case of a weak
lightsource as the moon or the atmosphere itself (thermal radiation). 

An absorbing layer is modeled by simply stacking cross sections. I got
the idea from the QPACK package (Erikson, 1999) but dont know if it
has been developped by them or used anywhere else before.

The absorbing layer is modeled as a polynomial, taking the
exponetially decreasing air pressure iinto account. 

\begin{eqnarray}
  p(\nu)_i = p(\nu)_{i-1} + a_i * \nu_{mid}
\end{eqnarray}



\end{document}
